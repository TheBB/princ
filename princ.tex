% {{{ Preamble
\documentclass[a4paper, twoside, notitlepage, 11pt]{article}

\usepackage[T1]{fontenc}
\usepackage[utf8]{inputenc}
\usepackage{lmodern}

\usepackage{amsmath,amsfonts,amssymb}
\usepackage{amsthm}
\usepackage{amscd}
\usepackage{bbm}
\usepackage{bm}
\usepackage{nicefrac}
\usepackage{mathtools}

\DeclareMathOperator{\sinc}{sinc}

\theoremstyle{plain}
\newtheorem{prop}{Proposition}
\newtheorem{lem}[prop]{Lemma}
\newtheorem{thm}[prop]{Theorem}
\newtheorem{cor}[prop]{Corollary}
\newtheorem{ass}[prop]{Assumption}
\newtheorem{ax}[prop]{Axiom}
\theoremstyle{definition}
\newtheorem{definition}[prop]{Definition}
\theoremstyle{remark}
\newtheorem{remark}[prop]{Remark}

\numberwithin{equation}{section}

% Number systems, fields, rings, groups...
\newcommand{\bbR}{\mathbb{R}}
\newcommand{\bbC}{\mathbb{C}}
\newcommand{\bbZ}{\mathbb{Z}} 
\newcommand{\bbN}{\mathbb{N}}
\newcommand{\bbS}{\mathbb{S}}

% Calligraphic style
\newcommand{\cA}{\mathcal{A}}
\newcommand{\cD}{\mathcal{D}}
\newcommand{\cT}{\mathcal{T}}
\newcommand{\cP}{\mathcal{P}}
\newcommand{\cB}{\mathcal{B}}
\newcommand{\cI}{\mathcal{I}}
\newcommand{\cO}{\mathcal{O}}

% Roman style
\newcommand{\rmK}{\mathrm{K}}
\newcommand{\rmM}{\mathrm{M}}
\newcommand{\rmP}{\mathrm{P}}
\newcommand{\rmS}{\mathrm{S}}

% Boldface
\newcommand{\Bc}{{\bm c}}
\newcommand{\Be}{{\bm e}}
\newcommand{\Bg}{{\bm g}}
\newcommand{\Bk}{{\bm k}}
\newcommand{\Bl}{{\bm l}}
\newcommand{\Bm}{{\bm m}}
\newcommand{\Bn}{{\bm m}}
\newcommand{\Bq}{{\bm q}}
\newcommand{\Bs}{{\bm s}}
\newcommand{\Bu}{{\bm u}}
\newcommand{\Bv}{{\bm v}}
\newcommand{\Bw}{{\bm w}}
\newcommand{\Bx}{{\bm x}}
\newcommand{\By}{{\bm y}}

\newcommand{\BA}{{\bm A}}

\newcommand{\Bxi}{{\bm\xi}}
\newcommand{\Bsig}{{\bm\sigma}}
\newcommand{\Bsigma}{\Bsig}
\newcommand{\Balpha}{{\bm\alpha}}
\newcommand{\Bbeta}{{\bm\beta}}

\newcommand{\Bzero}{{\bm0}}

% Symbols
\newcommand{\defeq}{\vcentcolon=}
\newcommand{\eqdef}{=\vcentcolon}


\title{Mathematics from scratch}
\author{E. Fonn}

\begin{document}

\maketitle
% }}}

The purpose of this document is to build some familiar mathematical structures from ``scratch'', by which we
mean the axioms of Zermelo-Fraenkel set theory (hereafter known as ZF). It is not our intention to be fully
rigorous, we target here mostly interested laypeople and amateur mathematicians.

% {{{ ZF set theory
\section{Zermelo-Fraenkel set theory}

A {\em set} is to be understood as a collection of objects, in the sense that for any given set $X$ and an
object $x$, we can decide whether or not $x$ belongs to $X$. In particular, sets do not count repetitions,
merely membership ($x$ cannot belong to $X$ ``more than once'').

The basic way to form a set is to merely list its elements, in the following way:
\[
    X = \{a,b,c,d\}.
\]
Here, $X$ is the set containing $a$, $b$, $c$ and $d$, and nothing else. This is not the only way to build
sets, however. Thanks to Axiom~\ref{ax:spec}

The nature of the ``objects'' $x$ is left unaddressed. For our purposes, all the {\em mathematical} objects we
consider (sets, functions, ordered pairs, etc.) can ultimately be defined in terms of sets (although we will
generally not {\em think} of them as such). Thus it is enough to consider only the case where objects are also
sets, and we will do so for the remainder of this paper. {\em Everything is a set!}

We write $x\in X$ to mean that $x$ belongs to the set $X$, and $x\not\in X$ if it does not. We also say that
two sets $X$ and $Y$ are {\em disjoint} if they contain no common elements. A set $Y$ is a {\em subset} of $X$
if, whenever $z\in Y$, we have $z\in X$ also. This is denoted by $Y \subseteq X$.

The axioms of ZF set theory describe what operations we are allowed to perform on sets without prior
justification. We will now consider them one by one.

\begin{ax}[Axiom of Existentionality] \label{ax:ex}
Let $X$ and $Y$ be sets. Assume for all sets $z$ that $z\in X$ if and only if $z\in Y$. Then $X=Y$.
\end{ax}

This axiom tells us when two sets $X$ and $Y$ are equal, namely when they contain precisely the same elements.
If so, then $X$ can be substituted for $Y$ anywhere.

\begin{ax}[Axiom of Regularity] \label{ax:reg}
Let $X$ be a non-empty set. Then there exists a $z\in X$ so that $z$ and $X$ are disjoint.
\end{ax}

The Axiom of Regularity implies that no set can be a member of itself. Indeed, let $X$ be a set and consider
the set $\{X\}$. Since it is nonempty, it must have a member disjoint from itself, but $X$ is its only member.
Thus, $X$ and $\{X\}$ have no common elements, and in particular $X\not\in X$.

\begin{ax}[Axiom schema of Specification] \label{ax:spec}
Let $X$ be a set, and $\phi$ any property that may or may not be satisfied by elements of $X$. Then there
exists a set containing those elements of $X$ that satisfy $\phi$.
\end{ax}

The formal notation for this set is $\left\{ x \in X \;|\; \phi \right\}$, to be read as ``the set of all $x$
in $X$ satisfying $\phi$''.

If at least one set $X$ exists, we can use Axiom~\ref{ax:spec} to define the empty set $\emptyset$ as
\[
    \emptyset \defeq \left\{ x \in X \;|\; x \in x \right\}
\]
since by Axiom~\ref{ax:reg}, no set $x$ may be a member of itself. The empty set has the property that for all
$x$, $x\not\in\emptyset$.

\begin{ax}[Axiom of Pairing] \label{ax:pair}
Let $x$ and $y$ be sets. Then there exists a set $Z$ such that $x\in Z$ and $y\in Z$.
\end{ax}

Note that $Z$ may be larger than $\{x,y\}$. However, once we have the set $Z$ from Axiom~\ref{ax:pair}, we can
use Axiom~\ref{ax:spec} to give the existence of $\{x,y\}$ as
\[
    \{x,y\} \defeq \{ z \in Z \;|\; z=x \vee z=y \}.
\]

\begin{ax}[Axiom of Union] \label{ax:union}
Let $X$ be a set. Then there is a set $Y$ that contains those sets which are members of some member of $X$:
\[
    x \in z \wedge z \in X \implies z \in Y.
\]
\end{ax}

Note that, again, the set $Y$ can be larger than {\em just} the members of members of $X$, but just like
previously, we can use the axiom of specification to cut away the extra. We denote this set as the {\em
union} of all members of $X$: 
\[
    Y \defeq \bigcup_{z\in X} z
\]
In particular we have the following special notation for unions of a finite number of sets. If
$X=\{z_1,z_2,\ldots,z_n\}$, then 
\[
    \bigcup_{z\in X} z \eqdef z_1 \cup z_2 \cup \cdots \cup z_n.
\]

\begin{ax}[Axiom of Infinity] \label{ax:inf}
Let the {\em successor} of a set $x$ be given as $S(x) \defeq x \cup \{x\}$. Then there exists a set $X$ such that
$\emptyset\in X$ and, whenever $x\in X$ we have $S(x)\in X$.
\end{ax}

Axiom~\ref{ax:inf} postulates the existence of an infinitely large set. In fact, as we will set later, this is
just the set of natural numbers. It can also be used as the set $X$ allowing us to define the empty set
$\emptyset$ using Axiom~\ref{ax:spec}.

\begin{ax}[Axiom of Power Set] \label{ax:pow}
Let $X$ be a set. Then there is a set containing all the subsets of $X$.
\end{ax}

This set is called the {\em power set} of $X$, and is denoted by $\cP(X)$\footnote{Like before, Axiom
\ref{ax:pow} does not by itself guarantee the existence of $\cP(X)$, only of a set that is at least as large.
We use Axiom \ref{ax:spec} to ``cut it down''.}.

These seven axioms will be largely sufficient for our purposes here. There are two omissions made:
\begin{itemize}
    \item The axiom schema of replacement, which tells us that if $f$ is a function on a domain $A$, then the
    range of $f$ is a subset of some set $B$.
    \item The axiom of choice (or any of its equivalent notions).
\end{itemize}
The reader will have noticed the pattern: the ZF axioms often guarantee the existence of sets that may be
larger than necessary, while Axiom~\ref{ax:spec} is the tool that allows us to reduce sets to contain
precisely what we need.

Let us now define intersections and differences of sets:
\[
    X \cap Y \defeq \{ z \in X \cup Y \;|\; z \in X \wedge z \in Y \}, \qquad
    X - Y \defeq \{ z \in X \;|\; z \not\in Y \}.
\]
% }}}

% {{{ Document end
\end{document}
% }}}
